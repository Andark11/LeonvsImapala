\chapter{Guía de Instalación y Ejecución}

\section{Requisitos del Sistema}

\subsection{Software Necesario}

\begin{itemize}
    \item \textbf{Python}: Versión 3.8 o superior
    \item \textbf{Sistema Operativo}: Linux, Windows 10/11, o macOS
    \item \textbf{Memoria RAM}: Mínimo 4 GB (recomendado 8 GB para entrenamiento largo)
    \item \textbf{Espacio en disco}: 100 MB para código y modelos
    \item \textbf{Git}: Para clonar el repositorio (opcional)
\end{itemize}

\subsection{Verificación de Python}

Abrir terminal y ejecutar:

\begin{lstlisting}[language=bash, caption=Verificar versión de Python]
python --version
# o alternativamente:
python3 --version
\end{lstlisting}

La salida debe mostrar Python 3.8.x o superior.

\section{Instalación Paso a Paso}

\subsection{Método 1: Clonar desde GitHub}

\begin{lstlisting}[language=bash, caption=Clonar repositorio]
# Clonar el repositorio
git clone https://github.com/Andark11/LeonvsImapala.git

# Ingresar al directorio
cd LeonvsImapala

# Verificar archivos
ls -la
\end{lstlisting}

\subsection{Método 2: Descarga Directa}

\begin{enumerate}
    \item Visitar: \url{https://github.com/Andark11/LeonvsImapala}
    \item Clic en \texttt{Code > Download ZIP}
    \item Extraer el archivo ZIP
    \item Abrir terminal en el directorio extraído
\end{enumerate}

\subsection{Dependencias}

El proyecto \textbf{no requiere instalación de paquetes externos}. Utiliza únicamente la biblioteca estándar de Python:

\begin{lstlisting}[language=bash, caption=Verificar instalación]
# No hay dependencias, pero se puede verificar Python
python -c "import math, json, random, time; print('OK')"
\end{lstlisting}

\section{Ejecución del Proyecto}

\subsection{Ejecutar el Programa Principal}

En cualquier sistema operativo:

\begin{lstlisting}[language=bash, caption=Ejecutar el programa]
# Desde el directorio del proyecto
python main.py

# O con Python 3 explícitamente
python3 main.py
\end{lstlisting}

\subsection{Menú Principal}

Al ejecutar \texttt{main.py}, se muestra el menú interactivo:

\begin{lstlisting}[caption=Menú principal del sistema]
==========================================================
=     SISTEMA LEÓN vs IMPALA - Q-LEARNING              =
==========================================================

1. Entrenar nuevo modelo
2. Cargar modelo existente
3. Ver base de conocimientos
4. Demostración paso a paso
5. Salir

Seleccione una opción:
\end{lstlisting}

\subsection{Opciones de Ejecución}

\subsubsection{Opción 1: Entrenar Nuevo Modelo}

\begin{lstlisting}[language=bash, caption=Flujo de entrenamiento]
Seleccione una opción: 1

Cuantos episodios desea entrenar? 1000
Nombre del modelo (sin extension): mi_modelo

[====================] 1000/1000 episodios
Exitos: 105 (10.5%)
Fracasos: 895 (89.5%)

Modelo guardado: modelos/mi_modelo.json
\end{lstlisting}

\subsubsection{Opción 2: Cargar Modelo Existente}

\begin{lstlisting}[language=bash, caption=Cargar modelo entrenado]
Seleccione una opción: 2

Modelos disponibles:
  1. EM4.json (100,000 episodios, 10.45% éxito)
  2. modelo_prueba.json (1,000 episodios, 8.2% éxito)

Seleccione modelo: 1

Modelo EM4 cargado exitosamente.
Q-Table: 15,234 estados
Base de conocimientos: 3,721 reglas
\end{lstlisting}

\subsubsection{Opción 3: Ver Base de Conocimientos}

\begin{lstlisting}[caption=Visualizar conocimientos adquiridos]
Seleccione una opción: 3

==========================================================
=          BASE DE CONOCIMIENTOS - EM4                 =
==========================================================

Conocimiento Específico (3,721 reglas):
-------------------------------------------
Regla #1:
  Condiciones: 
    - Distancia <= 2
    - Impala NO viendo al león
    - León NO escondido
  Acción recomendada: ATACAR
  Valor Q: +12.34
  Confianza: 94.2%

[... más reglas ...]

Conocimiento Generalizado (8 reglas clave):
-------------------------------------------
1. "Atacar cuando distancia < 2 y no detectado" (Q=+10.5)
2. "Esconderse cuando distancia > 7" (Q=+3.2)
3. "Avanzar en distancias medias (3-6)" (Q=+1.8)
[...]
\end{lstlisting}

\subsubsection{Opción 4: Demostración Paso a Paso}

\begin{lstlisting}[caption=Visualización paso a paso]
Seleccione una opción: 4

==========================================================
=           CACERIA PASO A PASO - EM4                  =
==========================================================

Turno 1:
---------
Estado:
  - León: Posición 1 (Norte), Distancia=9.5
  - Impala: Centro, Viendo FRENTE
  - León detectado: NO

[Grid 19x19]
    L
    .
    .
    I
    .

Decisión del león:
  Acción: AVANZAR (Q=+2.1)
  Razón: "Distancia lejana, acercarse sigilosamente"

[Presione Enter para siguiente turno...]
\end{lstlisting}

\section{Tests y Verificación}

\subsection{Ejecutar Tests Unitarios}

\begin{lstlisting}[language=bash, caption=Suite de tests]
cd tests
python test_basico.py -v

# Salida esperada:
test_abrevadero_coordenadas ... ok
test_leon_avanzar ... ok
test_impala_deteccion ... ok
test_q_learning_actualizacion ... ok
test_recompensas_calculo ... ok
test_caceria_completa ... ok

----------------------------------------------------------------------
Ran 6 tests in 0.234s

OK
\end{lstlisting}

\subsection{Verificar Estructura de Archivos}

\begin{lstlisting}[language=bash, caption=Script de verificación]
# Crear script verify.py
python -c "
import os
import sys

dirs = ['agents', 'simulation', 'knowledge', 
        'learning', 'storage', 'ui', 'tests']
files = ['main.py', 'environment.py', 'README.md']

for d in dirs:
    if not os.path.isdir(d):
        print(f'ERROR: Directorio {d} no encontrado')
        sys.exit(1)

for f in files:
    if not os.path.isfile(f):
        print(f'ERROR: Archivo {f} no encontrado')
        sys.exit(1)

print('✓ Estructura de archivos correcta')
"
\end{lstlisting}

\section{Configuración Avanzada}

\subsection{Ajustar Parámetros de Entrenamiento}

Editar \texttt{learning/q\_learning.py}:

\begin{lstlisting}[language=Python, caption=Parámetros personalizados]
class QLearning:
    def __init__(self, 
                 alpha: float = 0.05,      # Learning rate (0.01-0.1)
                 gamma: float = 0.9,       # Discount factor (0.8-0.99)
                 epsilon: float = 1.0,     # Exploration inicial (0.5-1.0)
                 epsilon_min: float = 0.1, # Exploration mínima (0.01-0.2)
                 epsilon_decay: float = 0.995):  # Decay (0.99-0.999)
        # ...
\end{lstlisting}

\subsection{Modificar Sistema de Recompensas}

Editar \texttt{learning/recompensas.py}:

\begin{lstlisting}[language=Python, caption=Ajustar pesos]
class SistemaRecompensas:
    EXITO_CACERIA = 100.0           # Aumentar/reducir según prioridad
    FRACASO_CACERIA = -50.0         # Penalización por fracaso
    ACERCAMIENTO = 1.0              # Reward por acercarse
    ALEJAMIENTO = -2.0              # Penalización por alejarse
    # ... más parámetros
\end{lstlisting}

\section{Solución de Problemas Comunes}

\subsection{Problema: Python no encontrado}

\begin{lstlisting}[language=bash]
# Solución: Usar python3 explícitamente
python3 main.py

# O agregar alias (Linux/macOS)
echo "alias python=python3" >> ~/.bashrc
source ~/.bashrc
\end{lstlisting}

\subsection{Problema: Permisos denegados (Linux/macOS)}

\begin{lstlisting}[language=bash]
chmod +x *.py *.sh
python main.py
\end{lstlisting}

\subsection{Problema: Módulo no encontrado}

\begin{lstlisting}[language=bash]
# Verificar que estás en el directorio correcto
pwd  # Debe mostrar .../LeonvsImapala

# Verificar estructura
ls agents/ simulation/ knowledge/
\end{lstlisting}

\subsection{Problema: Entrenamiento muy lento}

\begin{lstlisting}[language=bash]
# Solución 1: Reducir episodios
# En main.py: entrenar con 1,000 en lugar de 100,000

# Solución 2: Desactivar verbose
# En caceria.py: ejecutar_caceria_completa(verbose=False)
\end{lstlisting}

\section{Recursos Adicionales}

\subsection{Archivos de Referencia}

\begin{itemize}
    \item \texttt{README.md}: Documentación completa del proyecto
    \item \texttt{PRESENTACION\_5MIN.md}: Script de presentación
    \item \texttt{RESUMEN\_PROYECTO.md}: Resumen ejecutivo
    \item \texttt{ESTADO\_FINAL.txt}: Estado actual del desarrollo
\end{itemize}

\subsection{Enlaces Útiles}

\begin{itemize}
    \item Repositorio: \url{https://github.com/Andark11/LeonvsImapala}
    \item Documentación Python: \url{https://docs.python.org/3/}
    \item Q-Learning Tutorial: \url{https://en.wikipedia.org/wiki/Q-learning}
\end{itemize}

\subsection{Contacto y Soporte}

Para dudas o problemas:
\begin{itemize}
    \item GitHub Issues: \url{https://github.com/Andark11/LeonvsImapala/issues}
    \item Autores: Ver sección de autores en README.md
\end{itemize}
