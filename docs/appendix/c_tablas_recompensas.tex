\chapter{Tablas Completas de Recompensas}

\section{Tabla de Pesos Base}

\begin{table}[H]
\centering
\caption{Sistema completo de pesos de recompensa}
\label{tab:pesos-completo}
\begin{tabular}{|l|c|l|}
\hline
\textbf{Constante} & \textbf{Valor} & \textbf{Descripción} \\
\hline
\hline
\texttt{EXITO\_CACERIA} & +100.0 & León atrapa al impala exitosamente \\
\hline
\texttt{FRACASO\_CACERIA} & -50.0 & Impala logra huir del abrevadero \\
\hline
\texttt{ACERCAMIENTO} & +1.0/cuadro & León reduce distancia al impala \\
\hline
\texttt{ALEJAMIENTO} & -2.0/cuadro & León aumenta distancia al impala \\
\hline
\texttt{DETECCION\_TEMPRANA} & -5.0 & Impala detecta león (distancia 4-7) \\
\hline
\texttt{DETECCION\_MUY\_TEMPRANA} & -10.0 & Impala detecta león (distancia $>$7) \\
\hline
\texttt{TIEMPO\_EXCESIVO} & -0.1/turno & Penalización por turno consumido \\
\hline
\texttt{BUEN\_USO\_ESCONDERSE} & +2.0 & Esconderse cuando distancia $>$5 \\
\hline
\texttt{MAL\_USO\_ESCONDERSE} & -1.0 & Esconderse cuando distancia $<$3 \\
\hline
\texttt{ATAQUE\_CERCANO} & +5.0 & Atacar cuando distancia $\leq$2 \\
\hline
\texttt{ATAQUE\_LEJANO} & -3.0 & Atacar cuando distancia $>$3 \\
\hline
\end{tabular}
\end{table}

\section{Recompensas por Acción del León}

\subsection{AVANZAR}

\begin{table}[H]
\centering
\caption{Recompensas para acción AVANZAR según distancia}
\label{tab:avanzar-detalle}
\begin{tabular}{|c|c|l|}
\hline
\textbf{Distancia} & \textbf{Recompensa} & \textbf{Componentes} \\
\hline
\hline
$>$7 cuadros & +1.0 a +2.0 & ACERCAMIENTO \\
 & & Si detectado: -10.0 (DETECCION\_MUY\_TEMPRANA) \\
\hline
4-7 cuadros & +1.0 a +2.0 & ACERCAMIENTO \\
 & & Si detectado: -5.0 (DETECCION\_TEMPRANA) \\
\hline
2-3 cuadros & +1.0 a +1.5 & ACERCAMIENTO (acercamiento final) \\
\hline
$<$2 cuadros & +1.0 & ACERCAMIENTO \\
 & & Recomendación: cambiar a ATACAR \\
\hline
\end{tabular}
\end{table}

\textbf{Fórmula exacta:}
\begin{equation}
R_{\text{avanzar}} = \begin{cases}
\text{ACERCAMIENTO} \times \Delta d + \text{penalizaciones} & \text{si se acerca} \\
\text{ALEJAMIENTO} \times |\Delta d| & \text{si se aleja} \\
-0.1 & \text{por turno}
\end{cases}
\end{equation}

Donde $\Delta d = d_{\text{antes}} - d_{\text{después}}$

\subsection{ESCONDERSE}

\begin{table}[H]
\centering
\caption{Recompensas para acción ESCONDERSE según contexto}
\label{tab:esconderse-detalle}
\begin{tabular}{|c|c|c|l|}
\hline
\textbf{Distancia} & \textbf{Detectado} & \textbf{Recompensa} & \textbf{Razón} \\
\hline
\hline
$>$5 cuadros & NO & +2.0 & BUEN\_USO (sigilo efectivo) \\
\hline
$>$5 cuadros & SÍ & +0.5 & Uso tardío pero útil \\
\hline
3-5 cuadros & NO & +1.0 & Uso aceptable \\
\hline
3-5 cuadros & SÍ & -0.5 & Ya detectado, poco efectivo \\
\hline
$<$3 cuadros & NO & -1.0 & MAL\_USO (muy cerca, atacar mejor) \\
\hline
$<$3 cuadros & SÍ & -2.0 & Inútil: detectado y cerca \\
\hline
\end{tabular}
\end{table}

\textbf{Regla heurística:}
\begin{itemize}
    \item \textbf{Buena estrategia}: Esconderse en distancias $>$5 cuando NO detectado
    \item \textbf{Estrategia subóptima}: Esconderse ya detectado
    \item \textbf{Mal uso}: Esconderse a distancia $<$3 (momento de atacar)
\end{itemize}

\subsection{ATACAR}

\begin{table}[H]
\centering
\caption{Recompensas para acción ATACAR según distancia}
\label{tab:atacar-detalle}
\begin{tabular}{|c|c|c|l|}
\hline
\textbf{Distancia} & \textbf{Resultado} & \textbf{Recompensa} & \textbf{Descripción} \\
\hline
\hline
$\leq$1.5 cuadros & Éxito & +100.0 & EXITO\_CACERIA \\
 & & +5.0 & ATAQUE\_CERCANO (bonus) \\
 & & \textbf{Total: +105.0} & \textbf{Máxima recompensa} \\
\hline
1.5-2.0 cuadros & Éxito & +100.0 & EXITO\_CACERIA \\
 & & +5.0 & ATAQUE\_CERCANO \\
 & & \textbf{Total: +105.0} & Distancia óptima \\
\hline
2.0-3.0 cuadros & Fallido & 0.0 & Sin bonus ni penalización \\
 & & & (demasiado lejos para éxito) \\
\hline
$>$3.0 cuadros & Fallido & -3.0 & ATAQUE\_LEJANO \\
 & & & (ataque prematuro) \\
\hline
\end{tabular}
\end{table}

\textbf{Probabilidad de éxito:}
\begin{equation}
P_{\text{éxito}}(d) = \begin{cases}
95\% & d \leq 1.5 \\
70\% & 1.5 < d \leq 2.0 \\
20\% & 2.0 < d \leq 3.0 \\
0\% & d > 3.0
\end{cases}
\end{equation}

\subsection{SITUARSE}

\begin{table}[H]
\centering
\caption{Recompensas para acción SITUARSE}
\label{tab:situarse-detalle}
\begin{tabular}{|l|c|l|}
\hline
\textbf{Escenario} & \textbf{Recompensa} & \textbf{Descripción} \\
\hline
\hline
Posición mejorada & +0.5 & Mejor ángulo de ataque \\
\hline
Posición neutral & 0.0 & Sin cambio estratégico \\
\hline
Posición empeorada & -0.5 & Peor ángulo (más visible) \\
\hline
Tiempo consumido & -0.1 & Penalización por turno \\
\hline
\end{tabular}
\end{table}

\textbf{Nota}: Acción poco utilizada en modelos entrenados (menos del 5\% de acciones).

\section{Recompensas por Acción del Impala}

\subsection{VER\_IZQUIERDA / VER\_DERECHA / VER\_FRENTE}

\begin{table}[H]
\centering
\caption{Efecto de acciones de visión del impala}
\label{tab:impala-vision}
\begin{tabular}{|l|c|l|}
\hline
\textbf{Situación} & \textbf{Efecto en León} & \textbf{Mecanismo} \\
\hline
\hline
León en cono de visión & -5.0 a -10.0 & DETECCION\_TEMPRANA/MUY\_TEMPRANA \\
\hline
León fuera del cono & 0.0 & Sin detección \\
\hline
León escondido & 0.0 & No puede ser detectado \\
\hline
\end{tabular}
\end{table}

\textbf{Cono de visión:}
\begin{itemize}
    \item \textbf{Ángulo}: 120 grados (60° a cada lado de la dirección)
    \item \textbf{Alcance}: Ilimitado en el abrevadero
    \item \textbf{Efecto esconderse}: Anula detección en ese turno
\end{itemize}

\subsection{BEBER\_AGUA}

\begin{table}[H]
\centering
\caption{Efecto de acción BEBER\_AGUA}
\label{tab:impala-beber}
\begin{tabular}{|l|c|l|}
\hline
\textbf{Situación} & \textbf{Efecto} & \textbf{Descripción} \\
\hline
\hline
León cerca ($<$3) & Vulnerable & Impala no vigila \\
 & & León obtiene turno gratis \\
\hline
León lejos ($>$3) & Sin efecto & Acción segura \\
\hline
\end{tabular}
\end{table}

\subsection{HUIR}

\begin{table}[H]
\centering
\caption{Condiciones y efecto de HUIR}
\label{tab:impala-huir}
\begin{tabular}{|l|c|l|}
\hline
\textbf{Condición} & \textbf{Resultado} & \textbf{Recompensa León} \\
\hline
\hline
Distancia $<$3 y detectado & Huida exitosa & -50.0 (FRACASO\_CACERIA) \\
\hline
Distancia $\geq$3 & Huida exitosa & -50.0 (FRACASO\_CACERIA) \\
\hline
Distancia $<$3 y NO detectado & No huye & 0.0 (continúa cacería) \\
\hline
\end{tabular}
\end{table}

\section{Matriz de Recompensas Combinadas}

\begin{table}[H]
\centering
\caption{Matriz de recompensas León-Impala según distancia}
\label{tab:matriz-combinada}
\resizebox{\textwidth}{!}{%
\begin{tabular}{|c|c|c|c|c|}
\hline
\textbf{Distancia} & \textbf{AVANZAR} & \textbf{ESCONDERSE} & \textbf{ATACAR} & \textbf{SITUARSE} \\
\hline
\hline
$>$7 cuadros & +1.0 a +2.0 & +2.0 (si no detectado) & -3.0 (muy lejos) & ±0.5 \\
 & -10.0 si detectado & +0.5 (si detectado) & & \\
\hline
4-7 cuadros & +1.0 a +2.0 & +1.0 a +2.0 & -3.0 (lejos) & ±0.5 \\
 & -5.0 si detectado & & & \\
\hline
2-3 cuadros & +1.0 a +1.5 & -1.0 (mal uso) & 0.0 a -3.0 & ±0.5 \\
 & & & (bajo éxito) & \\
\hline
$<$2 cuadros & +1.0 & -1.0 a -2.0 & +105.0 (éxito) & ±0.5 \\
 & & (inútil) & o 0.0 (fallo) & \\
\hline
\end{tabular}
}
\end{table}

\section{Recompensas Acumuladas en Episodios Tipo}

\subsection{Episodio Exitoso (Estrategia Óptima)}

\begin{table}[H]
\centering
\caption{Ejemplo de recompensas en cacería exitosa}
\label{tab:episodio-exitoso}
\begin{tabular}{|c|l|c|c|}
\hline
\textbf{Turno} & \textbf{Acción} & \textbf{Recompensa} & \textbf{Acumulado} \\
\hline
\hline
1 & ESCONDERSE (dist=9.5) & +2.0 & +2.0 \\
2 & AVANZAR (9.5→8.5) & +1.0 & +3.0 \\
3 & AVANZAR (8.5→7.5) & +1.0 & +4.0 \\
4 & AVANZAR (7.5→6.5) & +1.0 & +5.0 \\
5 & AVANZAR (6.5→5.5) & +1.0 & +6.0 \\
6 & AVANZAR (5.5→4.5) & +1.0 & +7.0 \\
7 & AVANZAR (4.5→3.5) & +1.0 & +8.0 \\
8 & AVANZAR (3.5→2.5) & +1.0 & +9.0 \\
9 & AVANZAR (2.5→1.5) & +1.0 & +10.0 \\
10 & ATACAR (dist=1.5) & +105.0 & \textbf{+115.0} \\
\hline
\multicolumn{3}{|r|}{\textbf{Penalización tiempo}} & -1.0 \\
\hline
\multicolumn{3}{|r|}{\textbf{TOTAL FINAL}} & \textbf{+114.0} \\
\hline
\end{tabular}
\end{table}

\subsection{Episodio Fallido (Detección Temprana)}

\begin{table}[H]
\centering
\caption{Ejemplo de recompensas en cacería fallida}
\label{tab:episodio-fallido}
\begin{tabular}{|c|l|c|c|}
\hline
\textbf{Turno} & \textbf{Acción} & \textbf{Recompensa} & \textbf{Acumulado} \\
\hline
\hline
1 & AVANZAR (dist=9.5) & +1.0 & +1.0 \\
2 & AVANZAR (8.5→7.5) & +1.0 & +2.0 \\
3 & AVANZAR (detectado!) & -10.0 & -8.0 \\
4 & AVANZAR (persecución) & +1.0 & -7.0 \\
5 & AVANZAR & +1.0 & -6.0 \\
6 & ATACAR (dist=3.2, fallo) & -3.0 & -9.0 \\
7 & Impala HUYE & -50.0 & -59.0 \\
\hline
\multicolumn{3}{|r|}{\textbf{Penalización tiempo}} & -0.7 \\
\hline
\multicolumn{3}{|r|}{\textbf{TOTAL FINAL}} & \textbf{-59.7} \\
\hline
\end{tabular}
\end{table}

\section{Estadísticas de Recompensas (Modelo EM4)}

\begin{table}[H]
\centering
\caption{Distribución de recompensas en 100,000 episodios}
\label{tab:estadisticas-em4}
\begin{tabular}{|l|c|c|c|}
\hline
\textbf{Métrica} & \textbf{Éxitos} & \textbf{Fracasos} & \textbf{Promedio} \\
\hline
\hline
Recompensa final & +100 a +115 & -40 a -70 & +4.23 \\
\hline
Turnos promedio & 10.2 & 8.7 & 9.1 \\
\hline
Detecciones evitadas & 87\% & 23\% & 45\% \\
\hline
Uso de ESCONDERSE & 2.1 veces & 0.3 veces & 0.8 veces \\
\hline
Distancia ataque & 1.6 cuadros & 3.8 cuadros & 2.9 cuadros \\
\hline
\end{tabular}
\end{table}

\section{Comparación de Sistemas de Recompensas}

\begin{table}[H]
\centering
\caption{Comparación con sistemas alternativos probados}
\label{tab:comparacion-sistemas}
\begin{tabular}{|l|c|c|c|}
\hline
\textbf{Sistema} & \textbf{Tasa Éxito} & \textbf{Turnos Prom.} & \textbf{Nota} \\
\hline
\hline
Sistema Actual & 10.45\% & 9.1 & Balance óptimo \\
\hline
Solo Distancia & 3.2\% & 12.4 & No considera detección \\
\hline
Sin Detección Temprana & 6.7\% & 10.8 & Aprende más lento \\
\hline
Doble Penalización & 8.9\% & 8.3 & Muy conservador \\
\hline
Sin Tiempo & 9.1\% & 15.2 & Episodios muy largos \\
\hline
\end{tabular}
\end{table}

\textbf{Conclusión}: El sistema actual logra el mejor balance entre:
\begin{itemize}
    \item Tasa de éxito (10.45\%)
    \item Eficiencia temporal (9.1 turnos promedio)
    \item Aprendizaje de estrategias complejas (uso de ESCONDERSE)
\end{itemize}
